\documentclass[a4paper, oneside]{report}

\usepackage{helvet}
\renewcommand{\familydefault}{\sfdefault}

\input{preamble}
\input{macros}
\input{letterfonts}

\renewcommand{\baselinestretch}{1.2} 

\title{\Huge{Matemáticas 2}\\Problemario}
\author{\huge{Santiago de la cruz Martínez Lara}}
\date{Unknown XX, 2023}

\begin{document}

\maketitle
\newpage% or \cleardoublepage
% \pdfbookmark[<level>]{<title>}{<dest>}
\pdfbookmark[section]{\contentsname}{toc}
\tableofcontents
\pagebreak

\chapter{Segundo Parcial}
\section{Regla de L'Hôpital}

\qs{Calcula el siguiente limite utilizando la Regla de L'Hôpital cuando sea necesario}{
    $$\lim_{x \to 2}{\frac{5}{x^2+x-6} - \frac{1}{x-2}}$$
}

Al evaluar la función obtenemos el siguiente resultado:
$$\frac{5}{0} - \frac{1}{0}$$
Pero al no poderse aplicar la regla, tendremos que realiziar la operación antes de evaluar:
$$\lim_{x \to 2}{\frac{5}{x^2+x-6} - \frac{1}{x-2}}\equiv \lim_{x\to 2}\frac{4x-x^2-4}{x^3-x^2-8x + 12} = \frac{0}{0}$$\\
Siendo aplicable la Ley\\
$$\lim_{x \to 2}\frac{4x-x^2-4}{x^3-x^2-8x + 12} \equiv \lim_{x \to 2}{\frac{-2x+4}{3x^2-2x-8}} = \frac{0}{0} \therefore \\
\lim_{x \to 2}{\frac{-2x+4}{3x^2 -2x-8}} \equiv \lim_{x\to 2}{\frac{-2}{6x-2}} = -\frac{1}{5} $$

\sol{
    $$-\frac{1}{5}$$
}

\qs{Calcula el siguiente limite utilizando la Regla de L'Hôpital cuando sea necesario}{
    $$\lim_{x\to \frac{\pi}{2}}{\sec{x}-\tan{x}}$$
}
Debido a la naturaleza de estas funciones, obtenemos un valor indefinido al evaluar, por lo tanto tendremos que hacer alguna equivalencia:

$$\lim_{x\to \frac{\pi}{2}}{\sec{x} - \tan{x}} \equiv \lim_{x\to \frac{\pi}{2}}{\frac{1-\sin{x}}{\cos{x}} = \frac{0}{0} \therefore \lim_{x\to \frac{\pi}{2}}{\frac{1-\sin{x}}{\cos{x}} \equiv \lim_{x\to \frac{\pi}{2}}{\frac{-\cos{x}}{-\sin{x}}}=\frac{0}{1} = 0   $$
\\
\sol{0}

\qs{Calcula el siguiente limite utilizando la Regla de L'Hôpital cuando sea necesario}{
    $$\lim_{x\to 0}{\frac{\sin{5x}}{1-\cos{3x}}}$$
}
$$\lim_{x\to 0}{\frac{\sin{5x}}{1-\cos{3x}}} = \frac{0}{0} \therefore \equiv \lim_{x \to 0}{\frac{5\cos{5x}}{-\cos{3x}}} = -\frac{5}{1} = -5$$
\sol{-5}

\qs{Calcula el siguiente limite utilizando la Regla de L'Hôpital cuando sea necesario}{
    $$\lim_{x \to 0}{(1+\sin{4x})^{\cot{x}}}$$
}
Al evaluar la ecuación en 0, obtenemos $1^{\frac{1}{0}}$, debido a que esta ecuación no representa una división, tendremos que resolverla mediante el uso del logaritmo:\\
$$\lim_{x\to 0} \ln{y} = \cot{x}\times\ln{1+\sin{4x}}$$
\\Ahora al evaluar obtenemos $\infty \times 0$, por lo que tendremos que reescribir la ecuación para poder realizar la Regla de L'Hôpital:\\
$$\lim_{x\to 0}\ln{y} = \cot{x}\times\ln{1+\sin{4x}} \equiv \lim_{x\to 0}\ln{y}{\frac{\ln{1+\sin{4x}}}{\cot^{-1}{x}}} = \frac{1}{\frac{1}{0}}$$ $$\therefore$$ $$
\lim_{x\to 0}\ln{y}{\frac{\ln{1+\sin{4x}}}{\cot^{-1}{x}}} \equiv \lim_{x\to 0}{\ln{y}} = \frac{4\cos{4x}+4x^2\cos{4x}}{1+\sin{4x}} = 4 \rightarrow e^{4} $$
\sol{e^4}

\qs{Calcula el siguiente limite utilizando la Regla de L'Hôpital cuando sea necesario}{
    $$\lim_{x\to \frac{\pi}{2}}\frac{\tan{x}}{\tan{3x}}$$
}
Al evaluar la ecuación en $\frac{\pi}{2}$ obtenemos $\frac{0}{0}$, por lo que tendremos que descomponer la fracción:
$$\lim_{x\to \frac{\pi}{2}} \frac{\sin{x}\cos{3x}}{\sin{3x}\cos{x}} = \frac{0}{0}$$
$$\therefore$$
$$\lim_{x\to \frac{\pi}{2}} \frac{\sin{x}\cos{3x}}{\sin{3x}\cos{x}} \equiv \lim_{x\to \frac{\pi}{2}}\frac{2(\sin{4x}-\sin{2x})}{2(\sin{4x}+\sin{2x})} = \frac{0}{0} $$
$$\therefore $$
$$\lim_{x\to \frac{\pi}{2}}\frac{2(\sin{4x}-\sin{2x})}{2(\sin{4x}+\sin{2x})} \equiv \lim_{x\to \frac{\pi}{2}}\frac{4\cos{4x}-2\cos{2x}}{4\cos{4x}+2\cos{2x}} = \frac{6}{2} = 3 $$
\sol{3}

\qs{Calcula el siguiente limite utilizando la Regla de L'Hôpital cuando sea necesario}{}
1+1\\
\sol{2}

\qs{Calcula el siguiente limite utilizando la Regla de L'Hôpital cuando sea necesario}{}
1+1\\
\sol{2}

\qs{Calcula el siguiente limite utilizando la Regla de L'Hôpital cuando sea necesario}{}
1+1\\
\sol{2}

\qs{Calcula el siguiente limite utilizando la Regla de L'Hôpital cuando sea necesario}{}
1+1\\
\sol{2}

\qs{Calcula el siguiente limite utilizando la Regla de L'Hôpital cuando sea necesario}{}
1+1\\
\sol{2}


\pagebreak
\end{document}
